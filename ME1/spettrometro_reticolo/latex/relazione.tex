% !TeX root = relazione.tex
\documentclass{article}

\usepackage[utf8]{inputenc}
\usepackage[a4paper, total={15.3 cm, 21.3 cm}]{geometry}
\usepackage{amsmath}
\usepackage{amssymb}
\usepackage{gensymb}
\usepackage{booktabs}
\usepackage{hyperref}
\usepackage{caption}
\usepackage{float}
\usepackage{graphicx}
\usepackage{subfig}
\usepackage{titlesec}
\usepackage{titletoc}
\usepackage{physics}
\usepackage{siunitx}
\usepackage[dvipsnames]{xcolor}

\usepackage{longtable}
\usepackage{tabularx}
\usepackage{calc}
\usepackage{array}
\usepackage{subfiles} % Best loaded last in the preamble
\usepackage{etoolbox}
\usepackage{xparse}

\hypersetup{colorlinks=true,linkcolor=black}
\renewcommand\thesection{\arabic{section}}
\titlecontents{chapter}[1.05em]{\bigskip}
{\contentslabel[\MakeUppercase{\romannumeral\thecontentslabel}]{1em}\enspace\textsc}
{\hspace*{-1em}\textsc}
{\hfill\contentspage}
\titlecontents{section}[1.6em]{\smallskip}
{\thecontentslabel.\enspace}
{}
{\titlerule*[1pc]{.}\contentspage}
\setcounter{tocdepth}{2}


\begin{document}

    \pagenumbering{roman}
    \thispagestyle{empty}

    \begin{center}

        \includegraphics[width=1.\linewidth]{../../../tools/images/logo.jpg}
        \centering
        \vspace{3cm}

        \uppercase{\Large Relazione di laboratorio:\\ misura della velocità della luce\par}
        \vspace{3cm}

        \Large Lorenzo Liuzzo, Jiahao Miao, Riccardo Salto\par
        \vspace{1.5cm}

        \Large Novembre 23, 2022

    \end{center}
    \clearpage

    \tableofcontents
    \clearpage

    \pagenumbering{arabic}

    \section{Abstract}
    
        L'obbiettivo dell'esperienza di laboratorio è di misurare, tramite uno spettrometro a reticolo, 
        le lunghezze d'onda di alcune righe dello spettro di emissione di una sorgente di mercurio. \\

        Innanzitutto si è proceduto con la messa a fuoco dell'apparato e una prima ortogonalizzazione puntando il cannocchiale 
        su un obbiettivo sufficientemente lontano affinchè la luce entrasse ortogonale rispetto alla lente. 
        Dunque, è utilizzata una sorgente di sodio per ortogonalizzare il reticolo al fascio collimato attraverso la fenditura del collimatore, 
        la quale è stata regolata per massimizzare la luminosità del fascio. 
        Note le due lunghezze d'onda del doppietto del sodio e nota la simmetria degli ordini di interferenza rispetto all'ordine centrale, 
        l'ortogonalizzazione è stata effettuata attraverso la misura delle posizioni angolari 
        relative al medesimo ordine di interferenza di una riga del doppietto del sodio a sinistra e a destra dell'ordine centrale.
        Calcolando l'angolo di correzione, tramite la relazione \ref{eq:angolo di correzione},
        riposizionando lo strumento e ripetendo la misura, è stato possibile ottenere un fattore di correzione di $4'$, 
        inferiore alla precisione minima richiesta di $5'$. \\

        Successivamente, si è proceduto con la determinazione dei parametri caratteristici del reticolo: 
        la lunghezza $ L = 25 \pm 1 \, \mathrm{mm} $,
        il passo (formula \ref{eq:passo reticolo}) $ d = 3.4 \pm 0.7 \, \mathrm{\mu m} $, 
        il numero di fenditure (\ref{eq:numero di fenditure}) $ m = (75 \pm 1) \cdot \num{e2} $, 
        il potere dispersivo (\ref{eq:potere dispersivo}) $ D = (5.454 \pm 0.005) \cdot \num{e5} \, \mathrm{rad / m} $ e 
        il potere risolvente (\ref{eq:potere risolvente}) $ R = VALUE $. \\

        Infine, illuminando il reticolo con la lampada a mercurio si è proceduto con la misura delle posizioni angolari dei diversi ordini di interferenza
        e la misura indiretta delle lunghezze d'onda delle righe osservate. 
        Si riporta di seguito una tabella riassuntiva con i valori nominali noti e i risultati ottenuti. \\  

        \begin{table}[H]
            \centering
            \begin{tabular}{|c|c|c|c|}
                \toprule 
                colore & 
                $ \lambda_{\mathrm{nom}} \, \mathrm{[nm]} $ & 
                $ \lambda_{\mathrm{mis}} \, \mathrm{[nm]} $ & 
                $ \sigma \lambda_{\mathrm{mis}} \, \mathrm{[nm]}$ \\

                \midrule
                $ \mathrm{viola_{int}} $        & 404.7 & 404.0 & 0.4 \\
                $ \mathrm{viola_{est}} $        & 407.7 & 406.8 & 0.4 \\
                indaco                          & 435.8 & 434.8 & 0.4 \\
                verde $ \mathrm{acqua_{int}} $  & - & 490.9 & 0.5 \\  
                verde $ \mathrm{acqua_{est}} $  & - & 496.9 & 0.5 \\
                verde                           & 546.1 & 545.6 & 0.5 \\
                $ \mathrm{giallo_{int}} $       & 577.0 & 575.2 & 0.6 \\
                $ \mathrm{giallo_{est}} $       & 579.0 & 578.1 & 0.6 \\
                $ \mathrm{rosso_{int}} $        & - & 616 & 1 \\
                $ \mathrm{rosso_{est}} $        & - & 618 & 1 \\

                \bottomrule
            \end{tabular}
        \end{table}


    \section{Formule}

        \begin{equation}
            \label{eq:angolo di correzione}
            \beta = \frac{\theta_2 - \theta_1}{2} \cdot \frac{\cos(\theta_{av})}{1 - \cos\theta_{av}}
        \end{equation}

        \begin{equation}
            \label{eq:potere dispersivo}
            D = \frac{\Delta \phi}{\Delta \lambda} = \frac{k}{d \cos{\phi}}
        \end{equation}

        \begin{equation}
            \label{eq:potere risolvente}
            R = \frac{\lambda}{\Delta \lambda} = m \cdot k
        \end{equation}

        \begin{equation}
            \label{eq:passo reticolo}
            d = \frac{\lambda \cdot k}{\sin{\phi}} 
        \end{equation} 

        \begin{equation}
            \label{eq:numero di fenditure}
            m = \frac{L}{d}
        \end{equation}


    \section{Metodi}

        Tutte le misurazioni delle posizioni angolari dei massimi di interferenza sono state effettuate su un nonio con accuratezza di 1'. 
        Successivamente le misure sono state convertite in radianti. \\

        La prima misurazione effettuata è stata la misura della posizione angolare dell'ordine centrale, $\theta_0 = 0.9786	\pm 0.0003 \, \mathrm{rad}$. 
        L'incertezza sulla misura di $\theta_0$ è la precisione del nonio. \\

        In seguito, sono state misurate le posizioni angolari dei due primi massimi di interferenza di una riga del doppietto del sodio. 
        Sottraendo i due angoli misurati con $\theta_0$, calcolandone la media aritmetica e la discrepanza, si è calcolato il fattore di correzione $\beta$ 
        attraverso la relazione \ref{eq:angolo di correzione}. La procedura è stata ripetuta fino ad ottenere un valore di $beta = 4'$, sufficientemente piccolo
        per garantire un'ortogonalizzazione del reticolo al fascio collimato. \\ 

        Si è dunque proceduto con la misura della lunghezza del reticolo e delle posizioni angolari dei massimi di interferenza di tutte 
        le righe del doppietto del sodio e la determinazione dei parametri del reticolo. 
        Infatti, noto il passo del reticolo e misurate le posizioni angolari e i numeri di ordine di interferenza, invertendo la relazione \ref{eq:passo reticolo} è 
        possibile determinare le lunghezze d'onda $ \lambda $ di una differente sorgente luminosità. 
        In particolare, si è proceduto con la misura indiretta delle lunghezze d'onda delle righe osservate con la lampada a mercurio, il cui spettro presenta
        le seguenti lunghezze d'onda: viola int, viola est, indaco, verde acqua int, verde acqua est, verde, giallo int, giallo est, rosso int, rosso est. \\


    \section{Analisi dati}

        Per la determinazione della lunghezza d'onda è stata effettuata una media pesata sul set di dati raccolto per ogni colore, 
        mentre l'incertezza è stata calcolata in somma in quadratura con l'errore pesato e l'incertezza sulla misura del passo del reticolo. \\


    \section{Considerazioni finali}


\end{document}