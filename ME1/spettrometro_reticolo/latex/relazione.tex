% !TeX root = relazione.tex
\documentclass{article}

\usepackage[utf8]{inputenc}
\usepackage[a4paper, total={15.3 cm, 21.3 cm}]{geometry}
\usepackage{amsmath}
\usepackage{amssymb}
\usepackage{gensymb}
\usepackage{booktabs}
\usepackage{hyperref}
\usepackage{caption}
\usepackage{float}
\usepackage{graphicx}
\usepackage{subfig}
\usepackage{titlesec}
\usepackage{titletoc}
\usepackage{physics}
\usepackage{siunitx}
\usepackage[dvipsnames]{xcolor}

\usepackage{longtable}
\usepackage{tabularx}
\usepackage{calc}
\usepackage{array}
\usepackage{subfiles} % Best loaded last in the preamble
\usepackage{etoolbox}
\usepackage{xparse}

\hypersetup{colorlinks=true,linkcolor=black}
\renewcommand\thesection{\arabic{section}}
\titlecontents{chapter}[1.05em]{\bigskip}
{\contentslabel[\MakeUppercase{\romannumeral\thecontentslabel}]{1em}\enspace\textsc}
{\hspace*{-1em}\textsc}
{\hfill\contentspage}
\titlecontents{section}[1.6em]{\smallskip}
{\thecontentslabel.\enspace}
{}
{\titlerule*[1pc]{.}\contentspage}
\setcounter{tocdepth}{2}


\begin{document}

    \pagenumbering{roman}
    \thispagestyle{empty}

    \begin{center}

        \includegraphics[width=1.\linewidth]{../../../tools/images/logo.jpg}
        \centering
        \vspace{3cm}

        \uppercase{\Large Relazione di laboratorio:\\ misura della velocità della luce\par}
        \vspace{3cm}

        \Large Lorenzo Liuzzo, Jiahao Miao, Riccardo Salto\par
        \vspace{1.5cm}

        \Large Novembre 23, 2022

    \end{center}
    \clearpage

    \tableofcontents
    \clearpage

    \pagenumbering{arabic}

    \section{Abstract}
        L'obbiettivo dell'esperienza di laboratorio è di misurare, tramite uno spettrometro a reticolo, le lunghezze d'onda di alcune righe dello spettro di emissione di una sorgente di mercurio. \\
        Innanzitutto, è stata utilizzata una sorgente di sodio per ortogonalizzare il reticolo al fascio collimato dalla fenditura. 
        Infatti, note le due lunghezze d'onda del doppietto del sodio e nota la simmetria degli ordini di interferenza rispetto all'ordine centrale, 
        l'ortogonalizzazione è stata effettuata attraverso la misura delle posizioni angolari relative al medesimo ordine di interferenza di una riga del doppietto del sodio a sinistra e a destra dell'ordine centrale.
        Calcolando l'angolo di correzione, tramite la relazione \ref{eq:angolo di correzione}, riposizionando lo strumento e ripetendo la misura, è stato possibile ottenere un fattore di correzione di $3'$, inferiore alla precisione minima richiesta di $5'$. 
        Successivamente, si è proceduto con la determinazione dei parametri caratteristici del reticolo: il passo, il potere dispersivo e il potere risolvente: 
        $ d = 3.4 \pm 0.7 \, \mathrm{\mu m} $, $ D = (3.3 \pm 0.7) \cdot \num{e-5} \, \mathrm{\degree / m} $, $ R = VALUE $.

        Infine, illuminando il reticolo con la lampada a mercurio abbiamo misurato le posizioni angolari dei diversi ordini di interferenza. 
        Con tali valori, attraverso la relazione \ref{eq:lunghezza d'onda}, abbiamo determinato le lunghezze d'onda di emissione delle righe osservate. 
        Riportiamo di seguito una tabella riassuntiva dei risultati ottenuti, con anche i valori nominali noti. \\  

        \begin{table}[H]
            \centering
            \begin{tabular}{|c|c|c|c|c|}
                \toprule 
                colore & 
                $ \lambda_{\mathrm{nom}} \, \mathrm{[m]} $ & 
                $ \lambda_{\mathrm{mis}} \, \mathrm{[m]} $ & 
                $ \sigma \lambda_{\mathrm{mis}} \, \mathrm{[m]}$ & 
                $ \Delta \lambda \, \mathrm{[m]} $ \\

                \midrule
                $ \mathrm{viola_{int}} $        & 0 & 0 & 0 & 0 \\
                $ \mathrm{viola_{est}} $        & 0 & 0 & 0 & 0 \\
                indaco                          & 0 & 0 & 0 & 0 \\
                verde $ \mathrm{acqua_{int}} $  & 0 & 0 & 0 & 0 \\  
                verde $ \mathrm{acqua_{est}} $  & 0 & 0 & 0 & 0 \\
                verde                           & 0 & 0 & 0 & 0 \\
                $ \mathrm{giallo_{int}} $       & 0 & 0 & 0 & 0 \\
                $ \mathrm{giallo_{est}} $       & 0 & 0 & 0 & 0 \\
                $ \mathrm{rosso_{int}} $        & 0 & 0 & 0 & 0 \\
                $ \mathrm{rosso_{est}} $        & 0 & 0 & 0 & 0 \\

                \bottomrule
            \end{tabular}
        \end{table}

        
    \section{Ortogonalizzazione}


    \section{Formule}

        \begin{equation}
            \label{eq:angolo di correzione}
            \beta = \frac{\theta_2 - \theta_1}{2} \cdot \frac{\cos(\theta_{av})}{1 - \cos\theta_{av}}
        \end{equation}

        \begin{equation}
            \label{eq:potere dispersivo}
            D = \frac{\Delta \phi}{\Delta \lambda} = \frac{k}{d \cos{\phi}}
        \end{equation}

        \begin{equation}
            \label{eq:potere risolutivo}
            R = \frac{\lambda}{\Delta \lambda} = m \cdot k
        \end{equation}

        \begin{equation}
            \label{eq:passo reticolo}
            d = \frac{\lambda \cdot k}{\sin{\phi}} 
        \end{equation} 


    \section{Metodi}


    \section{Analisi dati}


    \section{Considerazioni finali}



\end{document}